\documentclass[a4paper,10pt]{article}

\usepackage{graphicx}
\usepackage{amsmath}
\usepackage{mathtools}
\usepackage[spanish]{babel}
\usepackage[utf8]{inputenc} % Permite escribir directamente áéíóúñ
\usepackage{float}
\usepackage[hidelinks]{hyperref}
\usepackage{pdfpages}
\usepackage{multirow}
\usepackage[margin=0.9in]{geometry}

\title{ \textbf{ 7529. Teoría de Algoritmos I\\
Trabajo Práctico 2}}

\author{ Ferreyra, Oscar, \textit{Padrón Nro. 89563} \\
\texttt{ fferreyra38@gmail.com } \\[2.5ex]
Martin, Débora, \textit{Padrón Nro. 90934} \\
\texttt{ demartin@fi.uba.ar } \\[2.5ex]
Eisner, Ariel, \textit{Padrón Nro. 90697} \\
\texttt{ aeeisnerg@gmail.com } \\[2.5ex]
\normalsize{2do. Cuatrimestre de 2016} \\
}

\date{}

\begin{document}
\maketitle
\thispagestyle{empty} % quita el nro en la primer pagina
\setcounter{page}{0}
\newpage
\tableofcontents

\newpage

\section{Clases de complejidad}


\subsection{Ciclos con peso negativo en un grafo Hamiltoniano}
Para la resolución de este problema se propone la utilización del Algoritmo de Floyd Warshall para encontrar el camino mínimo entre todos los vértices del grafo. Sobre la matriz resultante se analizan los números de la diagonal de la matriz de caminos. En caso de ser negativos, sabemos que pertenece a un ciclo negativo. Este algoritmo corre en tiempo \( O( V^{3}) \), siendo \( V \) la cantidad de vértices del grafo.

\bigskip

\textcolor[rgb]{0.4,0.4,0.4}{}\textcolor{black}{FloydWarshall() \{ }

\quad Armar la matriz de adyacencia F

\quad \textbf{\textcolor[rgb]{0.0,0.5019608,0.0}{for}}\textcolor{black}{ ( k from 1 to n )}

\quad \quad \textbf{\textcolor[rgb]{0.0,0.5019608,0.0}{for}}\textcolor{black}{ ( i from 1 to n )}

\quad \quad \quad \textbf{\textcolor[rgb]{0.0,0.5019608,0.0}{for}}\textcolor{black}{ ( j from 1 to n )}

\quad \quad \quad \quad F[i,j]=min(F[i,j], F[i,k] + F[k,j])

\quad \quad \quad \textbf{\textcolor[rgb]{0.0,0.5019608,0.0}{end for j}}

\quad \quad \textbf{\textcolor[rgb]{0.0,0.5019608,0.0}{end for i}}

\quad \textbf{\textcolor[rgb]{0.0,0.5019608,0.0}{end for k}}

\bigskip

\quad \textbf{\textcolor[rgb]{0.0,0.5019608,0.0}{for}}\textcolor{black}{ ( i from 1 to n )}

\quad \quad \textbf{\textcolor[rgb]{0.0,0.5019608,0.0}{if}}\textcolor{black}{ ( F[i,i] < 0 then )}
	
\quad \quad \quad print “Ciclo negativo”
		
\quad \quad \quad break
		
\quad \quad \textbf{\textcolor[rgb]{0.0,0.5019608,0.0}{end if}}
	
\quad \textbf{\textcolor[rgb]{0.0,0.5019608,0.0}{end for i}}

\}


\subsection{Ciclos de peso nulo en un grafo Hamiltoniano}

Este problema es NP-Completo.
 
Primero, dado un ciclo simple en el grafo G, podemos determinar si la suma de los pesos de sus vértices en tiempo polinomial, luego el problema de encontrar un ciclo de peso 0 es NP.

Luego, reducimos el problema de la suma de subconjuntos a este problema. Este problema es: dado un conjunto de enteros, ¿existe algún subconjunto cuya suma sea exactamente cero? A continuación describimos el algoritmo de resolución de dicho problema:

\bigskip

Consideramos un conjunto de enteros \( S = \{ a_{1}, . . . , a_{n} \}  \). Construimos un digrafo ponderado G con \( 2n \) vértices, para el cual cada elemento \( a_{i} \) corresponde con dos vértices \( v_{i} \) y \( u_{i} \)

\textbf{\textcolor[rgb]{0.0,0.5019608,0.0}{for each }}\( v_{i} \)

\quad agregar una arista de \( v_{i} \) en \( u_{i} \) con peso \( a_{i} \) y agregar aristas de cada vértice \( v_{j} \) en él con peso 0.

\textbf{\textcolor[rgb]{0.0,0.5019608,0.0}{for each}} \( u_{i} \)

\quad agregar aristas de éste vértice \( u_{i} \) a cada uno de los \( v_{j} \) con peso 0. 

\textbf{\textcolor[rgb]{0.0,0.5019608,0.0}{if}} (encontramos un ciclo de peso 0 en G) \textbf{\textcolor[rgb]{0.0,0.5019608,0.0}{then}}

\quad todos los pesos desde \( v_{j} \) hasta \( u_{j} \) a lo largo del ciclo deben ser cero.
		
\textbf{\textcolor[rgb]{0.0,0.5019608,0.0}{if}} (obtenemos un subconjunto \( S0 \subseteq S \) cuya sumatoria da 0)

\quad construimos un ciclo tomando todas las aristas \( (v_{j}, u_{j}) \) correspondientes al elemento en S0 y conectamos sus aristas a través de las aristas de ppeso cero, y finalmente obtenemos un ciclo de peso 0.

\bigskip

Por lo tanto, el problema es al menos tan difícil como el de suma de subconjuntos. Dado que éste es un problema NP-Completo, el problema de los ciclos de peso 0 también lo es.

\subsection{Tareas con ganancia mayor a un k dado}

\subsection{Tareas con ganancia igual a un k dado}


\section{Algoritmos de aproximación}


\subsection{El problema de la mochila}


\subsection{El problema del viajante de comercio}

Para implementar la solución aproximada a este problema se utilizó el algoritmo de Prim para calcular el árbol de tendido mínimo. La complejidad de dicho algoritmo depende de las operaciones de una cola de prioridad que contiene los vertices, de las iteraciones que se realizan sobre todas las aristas de cada vértice. Se tomará $|V|$ como la cantidad de vértices del grafo y $|A|$ como la cantidad de aristas. 
Las operaciones más importantes de la cola de prioridad son $O(\log{|V|})$ cada una, excepto el contains que es $O(|V|)$.
Por otro lado el ciclo principal es $O(|V|^2 *|A|)$ porque recorre para cada vertice, sus aristas, y verifica si su arista adyacente está en la cola de prioridad. El mencionado sería el peor caso, pero a medida que va avanzando la ejecución, la cola se va vaciando y el contains pasa a ser insignificante. Por último, se agregan todos los vertices y las aristas pertenecientes al árbol en un nuevo grafo. Esta operación es $O(|A| \log{|A|})$, en el peor caso que sería que todas las aristas del grafo original debieran ser agregadas al nuevo grafo. Pero en la mayoría de los casos será mucho menor y dependerá de la cantidad de aristas del árbol.

Además del algoritmo de Prim, luego se busca el preorder del árbol de tendido mínimo. Este recorre todos los vértices del grafo por lo que es $O(|V|)$.

Los tiempos de ejecución fueron llamativamente inferiores a los obtenidos por programación dinámica, aunque las soluciones obtenidas no resultaron siempre iguales a las anteriores. A continuación se muestran dichos tiempos de ejecución. Es de notar que en esta ocasión sí fue posible realizar la corrida para grafos con mayor cantidad de vertices.

\begin{table}[H]
\centering
\begin{tabular}{|c|c|c|c|c|c|c|}
\hline
Cantidad de ciudades	& Tiempo de ejecución (segundos)\\\hline
15						& $4.86 * 10^{-3}$\\\hline
17						& $5.74 * 10^{-3}$\\\hline
19						& $8.82 * 10^{-3}$\\\hline
21						& $6.35 * 10^{-3}$\\\hline
23						& $6.15 * 10^{-3}$\\\hline
48						& $13.4 * 10^{-3}$\\\hline
\end{tabular}
\caption{Tiempo en función de la cantidad de ciudades para la aproximación por el árbol de tendido mínimo}
\label{tab:held}
\end{table}


\end{document}
